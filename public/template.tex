% !TEX TS-program = xelatex
% !TEX encoding = UTF-8 Unicode 

% \documentclass[AutoFakeBold]{LZUThesis}
\documentclass[AutoFakeBold]{LZUThesis}
\usepackage{multirow}
\usepackage{threeparttable}
\CTEXsetup[name={第,部分}]{chapter}
\lstset{
language = MATLAB,
backgroundcolor=\color{white},   % choose the background color; you must add \usepackage{color} or \usepackage{xcolor}  
basicstyle=\footnotesize,        % the size of the fonts that are used for the code  
breakatwhitespace=false,         % sets if automatic breaks should only happen at whitespace  
breaklines=true,                 % sets automatic line breaking  
captionpos=bl,                    % sets the caption-position to bottom  
% commentstyle=\color{green},    % comment style  
% deletekeywords={...},            % if you want to delete keywords from the given language  
% escapeinside={\%*}{*)},          % if you want to add LaTeX within your code  
extendedchars=true,              % lets you use non-ASCII characters; for 8-bits encodings only, does not work with UTF-8  
frame=shadowbox,                    % adds a frame around the code  
keepspaces=true,                 % keeps spaces in text, useful for keeping indentation of code (possibly needs columns=flexible)  
keywordstyle=\color{blue},       % keyword style  
% language=Python,                 % the language of the code  
morekeywords={*,...},            % if you want to add more keywords to the set  
numbers=left,                    % where to put the line-numbers; possible values are (none, left, right)  
numbersep=5pt,                   % how far the line-numbers are from the code  
numberstyle=\tiny\color{gray}, % the style that is used for the line-numbers  
rulecolor=\color{black},         % if not set, the frame-color may be changed on line-breaks within not-black text (e.g. comments (green here))  
showspaces=false,                % show spaces everywhere adding particular underscores; it overrides 'showstringspaces'  
showstringspaces=false,          % underline spaces within strings only  
showtabs=false,                  % show tabs within strings adding particular underscores  
stepnumber=1,                    % the step between two line-numbers. If it's 1, each line will be numbered  
stringstyle=\color{orange},     % string literal style  
tabsize=2,                       % sets default tabsize to 2 spaces  
% title=signalAnalysis.m           % show the filename of files included with \lstinputlisting; also try caption instead of title  
}  

\begin{document}
%=====%
%
%封皮页填写内容
%
%=====%

% 标题样式 使用 \title{{}}; 使用时必须保证至少两个外侧括号
%  如: 短标题 \title{{第一行}},  
% 	      长标题 \title{{第一行}{第二行}}
%             超长标题\tiitle{{第一行}{...}{第N行}}

\title{{兰州大学学生会元旦公关活动方案}}



% 标题样式 使用 \entitle{{}}; 使用时必须保证至少两个外侧括号
%  如: 短标题 \entitle{{First row}},  
% 	      长标题 \entitle{{First row}{ Second row}}
%             超长标题\entitle{{First row}{...}{ Next N row}}
% 注意:  英文标题多行时 需要在开头加个空格 防止摘要标题处英语单词粘连.

\author{\CJKfontspec{楷体}李文涛}
\major{电子信息基地班}
\college{320200928101}
\grade{2020级}




\maketitle
\frontmatter

%中文摘要
\ZhAbstract{
    本文首先以AMI码为对比对象介绍$\mathrm{HDB_3}$的由来,简略介绍该编码方案的优缺点,
    接着利用MATLAB进行编程,完成对原数字信号序列进行$\mathrm{HDB_3}$编码
    以及后续对信号的频谱分析,对编码
    时频域分析进行谱图可视化,并同时分析其在其传输速率和频谱利用率上的特点。
    最后根据其频谱特性分析其应用价值。
}
{优缺点,MATLAB,时频域分析,应用分析}


%英文摘要
\EnAbstract{This paper introduces the advantages and advantages of $\mathrm{HDB_3}$, 
and then uses MATLAB to program,
 complete the $\mathrm{HDB_3}$ encoding of the original digital signal sequence and the subsequent spectrum analysis of the signal, 
 and the spectral diagram visualization of the encoding time frequency domain analysis, 
 and analyze its characteristics in its transmission rate and frequency utilization.
 Finally, its application value is analyzed according to its spectral characteristics.
    \fontspec{Times New Roman}}
{Advantages and disadvantages, time frequency domain analysis, application analysis}

%生成目录
% \tableofcontents
% \addcontentsline{toc}{chapter}{目录}
% \thispagestyle{empty}


%文章主体
\mainmatter

\chapter{兰州大学学生会元旦公关活动方案可行性分析报告}

据悉,由于疫情影响,我校榆中校区同学们对于生活的调剂需求问题愈加尖锐。
榆中周边放松休闲的设施较少,一般来说,同学们都是抽出时间乘校车
去往城关区进行放松娱乐,而受疫情影响,校区采用封闭式管理,同学们
乘校车去往城关变得困难,由此导致同学们压力积攒。为了改善同学们的生活调剂
状况,作为同学为主体的学生会应该做出努力,为同学们的生活提供有效的调剂活动。

\section{地理位置}

学生会组织可以最大程度地利用学校所提供的场所进行活动,如视野广场、昆仑堂门前、南区浴室,
以上区域能够承载足够的人流量,可供同学们参加开展的活动
\chapter{计算结果分析}

\section{仿真数据分析}

\subsection{时域分布}

\begin{equation}
    \omega _i=2sin(\frac{i\pi}{2(N+1)})
\end{equation}

\begin{equation}
    \sum_{n = 1}^{\infty}  
\end{equation}

\begin{equation}
    H=\sum_{i = 1}^{N}  \frac{\dot{x_i}^2 }{2}+\sum_{i=0}^{N}[\frac{1}{2}k(x_{i+1}-x_i)^2+\frac{\alpha}{3}(x_{i+1}-x_i)^3+\frac{\beta}{4}(x_{i+1}-x_i)^4]
\end{equation}

\subsection{频域分布}

\section{优缺点分析}



\backmatter


% %=======%
% %引入参考文献文件
% %=======%
\bibdatabase{bib/POC}%bib文件名称 仅修改bib/ 后部分
\printbib
\nocite{*} %显示数据库中有的,但是正文没有引用的文献


% \Appendix

% 这里是附录页,可要可不要

% \Thanks.



\end{document}